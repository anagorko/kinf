\documentclass[a4paper, 11pt]{scrartcl}

\usepackage[utf8]{inputenc}
\usepackage[T1]{fontenc}
\usepackage[polish]{babel}
\usepackage{url}

\newcounter{zad}
\newcounter{pkt}

\newcommand{\zad}{\vspace{4mm}\noindent\setcounter{pkt}{0}\stepcounter{zad}{\bf \thezad.} }
\newcommand{\zadgw}{\vspace{4mm}\noindent\setcounter{pkt}{0}\stepcounter{zad}{\bf \thezad$^*$.} }
\newcommand{\pkt}{\vspace{1mm}\stepcounter{pkt}\alph{pkt}) }

\begin{document}

\title{Przetwarzanie tekstu. Klasa {\tt string}.}
\subtitle{Kółko Informatyczne szkoły Żagle}
\date{7 września 2012}
\maketitle

\section*{W pigułce}

{\tt string, [], .length()}, kody ASCII, tablice indeksowane znakami (kubełki)

\section*{Ćwiczenia na zajęcia i do domu}

\zad Napisz program, który pyta użytkownika o imię i na podstawie ostatniej litery imienia zgaduje jego płeć.

\zad Napisz program sprawdzający hasło (hasłem jest oczywiście "mellon").

\zad Napisz program szyfrujący tekst szyfrem Cezara.

\zad Napisz program, który zamienia wielkość liter w tekście (małe na wielkie, wielkie na małe - alfabet angielski).

\zad Napisz program zliczający samogłoski (łączna liczba).

\zad Napisz program zliczający samogłoski (każdą osobno).

\zad Napisz program wypisujący najczęściej występującą w tekście literę.

\zad Napisz program zamieniający samogłoski na literę z.

\zad Napisz program wypisujący każde słowo tekstu od tyłu.

%\zadgw Jak wyżej, ale nie odwracając zbitek "{}cz, sz, rz, ch, dź, dzi, si, ci".

\zad Napisz program zamieniający kolejność liter wewnątrz wyrazu, ale nie zmieniający pierwszej i ostatniej litery.

%\zadgw Jak wyżej, ale nie rozbijając zbitek "{}cz, sz, rz, ch, dź, dzi, si, ci".

\zad Napisz program sprawdzający, czy podane słowo jest palindromem (odp.: TAK lub NIE).

\zad Napisz program znajdujący i wypisujący wszystkie liczby z tekstu.

\zad Napisz program zwiększający daną liczbę o jeden (liczba może mieć do 1000 cyfr, układ dziesiętny).

\vspace{5mm}
\begin{center}
{\bf Podczas zajęć upewnijcie się, że będziecie w domu umieli wykonać wszystkie ćwiczenia!}
\end{center}

\section*{Do poczytania}

Ze strony

\vspace{5mm}
\url{http://cpp0x.pl/kursy/Kurs-C++/Poziom-3/346}
\vspace{5mm}

\noindent
przeczytajcie lekcje

\begin{enumerate}
\item[23.] Zmienne przechowujące tekst,
\item[26.] Wczytywanie tekstu - standardowy strumień wejścia,
\end{enumerate}

\noindent
i zróbcie ćwiczenia, które znajdziecie na końcu każdej z nich.

\section*{I Konkurs Szkolny, 7-13 września 2012}

%Sortowanie biżuterii (I OIG)
%Naszyjniki (III OIG)
%Bałagan i Stonki (PA 05.2002)

\noindent
Treści siedmiu zadań konkursowych znajdziecie pod adresem

\vspace{5mm}
\url{http://z1.sioportal.dasie.mimuw.edu.pl/index.php}

\vspace{5mm}
\noindent
Wszystkie zadania pochodzą z konkursów z zeszłych lat (PA i OIG).
Rozwiązania można wysyłać do godz. 19:59:59, 13 września. O godzinie 20:00:00 zostanie opublikowany ranking. Zadania możecie wykonywać w parach (programista + matematyk).

\vspace{5mm}
\begin{center}
{\bf Udział w konkursie jest obowiązkowy :)}
\end{center}

%Kamil (PA 2003)
%Dysleksja [B] (PA 2007)
%Jasio (PA 2005)
%Szyfr (III OIG) dobre
%Imiona mrówek (PA 05.2002)
%Statystyki (III OIG) dobre
%Zliczacz liter (kubełki) (I OIG)


\end{document}
