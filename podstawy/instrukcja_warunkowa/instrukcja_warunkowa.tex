\documentclass[a4paper, 10pt, pdftex]{amsart}

\usepackage{graphicx}
\usepackage[utf8]{inputenc}
\usepackage[T1]{fontenc}
\usepackage{url}
\usepackage{cmbright}

\newcounter{zad}
\newcounter{pkt}

\newcommand{\zad}{\vspace{4mm}\noindent\setcounter{pkt}{0}\stepcounter{zad}{\bf \thezad.} }
\newcommand{\zadgw}{\vspace{4mm}\noindent\setcounter{pkt}{0}\stepcounter{zad}{\bf \thezad$^*$} }
\newcommand{\pkt}{\vspace{1mm}\stepcounter{pkt}\alph{pkt}) }

\newcommand{\js}{{\tt js}}

\addtolength{\textheight}{4cm}

\begin{document}

\thispagestyle{empty}

\vspace*{-8mm}
\noindent{\Large\bf Instrukcja warunkowa}

\vspace{2mm}

\noindent{\large {\bf Instrukcja} {\tt if ( ... ) \{ ... \} else \{ ... \} }

\vspace{5mm}
\noindent
{\bf Wyrażenia logiczne}

\begin{enumerate}
\item {\tt ==} równość
\item {\tt !=} różne
\item {\tt >=} większe równe
\item {\tt >} większe
\item {\tt <} mniejsze
\item {\tt <=} mniejsze równe
\item {\tt \&\&} "i"
\item {\tt ||} "lub"
\item {\tt !} zaprzeczenie
\end{enumerate}

\vspace{2mm}\noindent
{\bf Przykład.} Wyrażenie logiczne "{\tt x != 7}" jest prawdziwe jeżeli x jest różne od $7$.

\vspace{2mm}\noindent
{\bf Przykład.} Wyrażenie logiczne "{\tt x >  0 \&\& x < 10}" jest prawdziwe gdy x jest większe  od zera i mniejsze od dziesięciu.

\vspace{2mm}\noindent
{\bf Przykład.} Wyrażenie logiczne "{\tt !(x >  0 \&\& x < 10)}" jest prawdziwe jeżeli poprzednie jest
nieprawdziwe, tzn. wtedy gdy x jest
  mniejsze bądź równe $0$ lub x jest większe bądź równe $10$.

\vspace{3mm}\noindent
{\bf Instrukcje warunkowe}

\zad Co wypisze program:
\vspace{1mm}
\begin{verbatim}
int t = 7;
if (t < 0) { 
  cout << "Jest mróz.");
} else {
  cout << "Jest odwilż.";
}
\end{verbatim}

\zad Co wypisze poniższy program dla $t = 2$, $t = 10$, $t = 15$ i $t = 35$?

\begin{verbatim}
if (t < 5) { 
  cout << "Załóż czapkę. ";
} else if (t > 30) {
  cout << "Załóż kapelusz słoneczny. ";
}
if (t > 20) {
  cout << "Załóż t-shirt.";
else if (t  <= 20 && t > 10) {
  cout << "Załóż sweter.";
} else {
  cout << "Załóż kurtkę.";
}

\end{verbatim}


\newpage\thispagestyle{empty}

\zad Poniższy program przeprowadza test z matematyki.
Stwórz podobny test ze swojego ulubionego przedmiotu.

\vspace{5mm}
\begin{verbatim}
	int w = 0; // wynik
	int o; // odpowiedź
	
	cout << "Jaki jest następny wyraz ciągu 1,1,2,3,5,8,13?";
	cin >> o;
	
	if (o == 21) {
	  w = w + 1;
	}
	
	cout << "Ile trójkątów równobocznych można utworzyć z sześciu zapałek?";
	cin >> o;
	
	if (o >= 4) {
	  w = w + 1;
	}

	cout << "Ile jest równe 341*275? Możesz pomylić się o 10.";
	cin >> o;
	
	if (o <= 341 * 275 + 10 && o >= 341 * 275 - 10) {
	  w = w + 1;
	}
	
	cout << "Uzyskałeś " << w << " punkt";
	
	if (w == 0) { cout << "ów."; }
	if (w == 1) { cout << "."; }
	if (w >= 2) { cout << "y."; }
\end{verbatim}

\end{document}
