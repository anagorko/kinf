\documentclass[a4paper, 10pt, pdftex, dvipsnames]{scrartcl}

\usepackage{graphicx}
\usepackage[utf8]{inputenc}
\usepackage[T1]{fontenc}
\usepackage{url}
\usepackage{cmbright}
\usepackage[polish]{babel}
\usepackage{tikz}

\newcounter{zad}
\newcounter{pkt}

\newcommand{\zad}{\vspace{4mm}\noindent\setcounter{pkt}{0}\stepcounter{zad}{\bf \thezad.} }
\newcommand{\zadgw}{\vspace{4mm}\noindent\setcounter{pkt}{0}\stepcounter{zad}{\bf \thezad$^*$} }
\newcommand{\pkt}{\vspace{1mm}\stepcounter{pkt}\alph{pkt}) }

\newcommand{\js}{{\tt js}}

\addtolength{\textheight}{4cm}


\tikzstyle{wired}=[draw=gray!30, line width=0.15mm]
\tikzstyle{number}=[anchor=center, color=white]
% Sectors are numbered 0-19 counterclockwise from the top.

% \strip{color}{sector}{outer_radius}{inner_radius}
\newcommand{\strip}[4]{
    \filldraw[#1, wired]
      ({18 *  #2}      :                   #3) arc
      ({18 *  #2}      : {18 * (#2 + 1)} : #3) --
      ({18 * (#2 + 1)} :                   #4) arc
      ({18 * (#2 + 1)} : {18 *  #2}      : #4) -- cycle;
}

% \sector{color}{sector}{radius}
\newcommand{\sector}[3]{
    \filldraw[#1, wired]
      (0, 0) --
      ({18 * #2} :                   #3) arc
      ({18 * #2} : {18 * (#2 + 1)} : #3) -- cycle;
}

\begin{document}

\section*{Pętle - 7 XII 2012}

\subsection*{Przypomnienie}

,,Wykonaj'' na kartce program:

\begin{verbatim}
  int a = 1;
  for (int i = 1; i <= 4; i++) {
    if ((i % 2) == 0) {
      a = a * i;
    } else {
      a = a + i;
    }
  }
  cout << "a = " << a << endl;
\end{verbatim}

\subsection*{Szyfr Cezara}

Jest to szyfr za pomocą którego Juliusz Cezar szyfrował swoje listy do Cycerona. Jako ciekawostkę można podać, że szyfr ten był podobno używany jeszcze w 1915 roku w armii rosyjskiej, gdyż tylko tak prosty szyfr wydawał się zrozumiały dla sztabowców\footnote{\url{http://www.kryptografia.com/algorytmy/cezar.html}}.

\vspace{2mm}\noindent
{\bf Wersja podstawowa:} Każdą literę tekstu jawnego zamieniamy na literę przesuniętą o 3 miejsca w prawo. I tak literę A szyfrujemy jako literę D, literę B jako E itd. W przypadku litery Z wybieramy literę C. W celu odszyfrowania tekst powtarzamy operację tym razem przesuwając litery o 3 pozycje w lewo.

\begin{center}
% 81 degrees = 4.5 sectors.
% The rotation leaves 20 at the top.
\begin{tikzpicture}[rotate=81, scale=.14]

  % These are the official dartboard dimensions as per BDO's regulations.

  % The whole board's background.
  \fill[black] (0, 0) circle (225.5mm);
  \fill[white] (0, 0) circle (125.5mm);

  % Labels.
  \foreach \sector/\label in {%
      1/A,  2/B, 3/C, 4/D, 5/E, 6/F, 7/G, 8/H, 9/I, 10/J, 11/K, 12/L, 13/M, 14/N, 15/O, 16/P, 17/Q, 18/R, 19/S, 20/T, 21/U, 22/V, 23/W, 24/X, 25/Y, 26/Z}%
  {
    \node[number, rotate={13.846 * (-\sector + 1)}] at ({13.846 * (-\sector + 1.5)} : 197.75mm) {\label};
    \node[number, font=\small, rotate={13.846 * (-\sector + 1)}] at ({13.846 * (-\sector + 1.5)} : 157.75mm) {\sector};
  }
\end{tikzpicture}
\end{center}

\vspace{2mm}\noindent
{\bf Przykład:} szyfrujemy zdanie „Lubię szkołę”. Po pierwsze – w naszym alfabecie (patrz rysunek) nie ma wielkich liter, polskich znaków i spacji. Dlatego będziemy szyfrować tekst „lubieszkole”. Przesunięcia odczytujemy z tabeli: l $\rightarrow$ o, u $\rightarrow$ x, b $\rightarrow$ e, i $\rightarrow$ l, e $\rightarrow$ h, s $\rightarrow$ v, z $\rightarrow$ c, k $\rightarrow$ n, o $\rightarrow$ r, l $\rightarrow$ o, e $\rightarrow$ h. Otrzymujemy napis „oxelhvcnroh”.
Zauważ, w jaki sposób zaszyfrowaliśmy literę z: alfabet się „zawinął”.

\vspace{2mm}\noindent
{\bf Ćwiczenie:} zaszyfruj swoje imię i nazwisko za pomocą szyfru Cezara.

\subsection*{Zadania}

W rozwiązaniu zadań mogą być pomocne przykładowe programy z zeszłego tygodnia: ,,suma liczb'' i ,,flaga polska'', zamieszczone na następnej kartce.

\subsubsection*{Obliczenia}

\zad Napisz program, który wyświtela na ekranie liczby od $100$ do $1$.

\zad Napisz program, który wyświetla na ekranie najpierw liczby nieparzyste od $1$ do $99$, a następnie liczby parzyste od $2$ do $100$.

\zad Napisz program, który oblicza sumę liczb parzystych od $2$ do $100$.

\zad Napisz program, który wyświetla liczby od $1$ do $100$, ale każdą liczbę podzielną przez $3$ zastępując tekstem ,,foo''.

\zad Napisz program, który wyświetla liczby od $1$ do $100$, zamiast liczb podzielnych przez $3$ wypisując ,,foo'', zamiast liczby podzielnych przez $3$ ale nie podzielnych przez $5$ wypisując ,,bar'' a zamiast liczb podzielnych przez $3$ i przez $5$ wypisując ,,foobar''.

\subsubsection*{Przetwarzanie tekstu}

Jeżeli {\tt s} jest zmienną typu {\tt string}, to

\begin{itemize}
  \item s.length() to długość napisu,
  \item s[0], s[1], s[2] to kolejne litery napisu.
\end{itemize}

\vspace{2mm}\noindent
{\bf Przykład:} 
\begin{verbatim}
  string napis = "To jest napis!";

  cout << napis.length() << endl;                // wypisana zostanie liczba 14
  cout << napis[3] << endl;                      // wypisana zostanie litera 'j'
  cout << napis[napis.length() - 1] << endl;     // wypisany zostanie znak '!'  
\end{verbatim}


\zadgw Napisz program, który pyta użytkownika o imię, a potem odpowiada ,,Jesteś chłopakiem!'' lub ,,Jesteś dziewczyną!''.

\zad Napisz program, który pyta użytkownika o wyraz, a następnie wyświetla ten wyraz od końca.

\zad Napisz program, który pyta użytkownika o wyraz, a następnie wyświetla go w następujący sposób: najpierw jego parzyste litery, a następnie nieparzyste.



\zad Napisz program, który szyfruje dany wyraz za pomocą szyfru Cezara.

\zad Napisz program, który odszyfrowuje dany wyraz zaszyfrowany szyfrem Cezara.

\subsection*{Suma liczb od $1$ do $100$}
\begin{verbatim}
  int suma = 0;
  for (int liczba = 1; liczba <= 100; liczba++) {
    suma = suma + liczba;
  }
  cout << "Suma liczb od 1 do 100 wynosi " << suma << "." << endl;
\end{verbatim}

\subsection*{Flaga polska}
\begin{verbatim}
  int s; cout << "Szerokość?"; cin >> s;
  int w; cout << "Wysokość?"; cin >> w;
  for (int i = 0; i < w/2; i++) {
    for (int j = 0; j < s; j++) { cout << '.'; }
    cout << endl;
  }
  for (int i = 0; i < w/2; i++) {
    for (int j = 0; j < s; j++) { cout << '#'; }
    cout << endl;
  }  
\end{verbatim}

\end{document}
