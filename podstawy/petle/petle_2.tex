\documentclass[a4paper,10pt]{scrartcl}

\usepackage{amsmath, amssymb}
\usepackage{tikz}
\usetikzlibrary{shapes,arrows}

\usepackage[utf8]{inputenc}
\usepackage[T1]{fontenc}
\usepackage{url}
\usepackage[polish]{babel}

\usepackage{listings}
\usepackage{xcolor}
\lstset { %
    language=C++,
    backgroundcolor=\color{black!5}, % set backgroundcolor
    basicstyle=\small\ttfamily
}

\newcounter{seria}
\newcounter{zad}
\newcounter{pkt}
\newcounter{totalcount}

\newcommand{\df}{\vspace{4mm}\noindent{\bf Definicja.} }
\newcommand{\wsk}{\vspace{4mm}\noindent{\bf Wskazówka.} }

\newcommand{\zad}{\stepcounter{totalcount}\vspace{4mm}\noindent\setcounter{pkt}{0}\refstepcounter{zad}{\bf \Roman{seria}.\thezad.} }
\newcommand{\zadgw}{\vspace{4mm}\noindent\setcounter{pkt}{0}\refstepcounter{zad}{\bf \thezad$^*$.} }
\newcommand{\pkt}{\vspace{1mm}\stepcounter{pkt}\alph{pkt}) }

\newcommand{\reset}{\setcounter{zad}{0}\setcounter{pkt}{0}\stepcounter{seria}}

\begin{document}

\vspace*{-1cm}
\pagestyle{myheadings}
\markright{{\tiny Kółko Matematyczno-Informatyczne Szkoły żagle \hfill Styczeń 2013}}

\reset
\section*{Grupa młodsza. Pętle}

Sponsorem dzisiejszych zajęć jest instrukcja {\tt while}, literka {\tt x} i liczba {\tt 3}.

\subsection*{Rozgrzewka bez komputera}

\zad Wykonaj następujący program dla $x = 3$, $x = 4$ i $x = 7$ (na tablicy).

\begin{lstlisting}
 while (x != 1) 
 {
    if (x % 2 == 0) {
      x = x / 2;
    } else {
      x = 3 * x + 1;
    }
    cout << x << endl;
  }
\end{lstlisting}

\zad Wyjaśnij działanie następującego programu ({\tt b} i {\tt c} to zmienne typu {\tt int}).

\begin{lstlisting}
int a = 0;
while (true)
{
    if (b == 0) break;
    if (b % 2 == 1) {  a += c; }
    b /= 2;
    c *= 2;
}
cout << a << endl;
\end{lstlisting}

\vspace{3mm}\noindent
{\bf Wskazówka.} Wykonaj program dla kilku wartości $b$ i $c$.

\vspace{3mm}\noindent
{\bf Wskazówka.} Pomyśl o rozkładzie dwójkowym liczby $b$.

\zad Wyjaśnij, co robi następujący program ({\tt n} to zmienna typu {\tt int}).

\begin{lstlisting}
int a = 1;
for (; n > 0; a *= n--);
cout << a << endl;
\end{lstlisting}

\zad Wyjaśnij różnicę pomiędzy programami
\begin{lstlisting}
while (n > 0) {
  cout << "Czesc!" << endl;
  n--;
}
\end{lstlisting}
\begin{lstlisting}
do {
  cout << "Czesc!" << endl;
  n--;
} while (n > 0);
\end{lstlisting}

\zad Wyjaśnij działanie programu
\begin{lstlisting}
for (int i = 0; i <= 10; i++) 
{
    if (i % 2 != 0) 
      continue;
      
    cout << i << endl;
}
\end{lstlisting}

\subsection*{Zadania na zajęcia}

\zad Napisz program wyświetlający $n$ razy napis ,,Witamy na pokładzie!'', za pomocą

\pkt pętli {\tt for}

\pkt pętli {\tt while}

\pkt pętli {\tt do \ldots while }

\zad Napisz program, który pyta o liczbę $n$ a następnie o $n$ liczb, których sumę wyświetla na ekranie.

\zad Napisz program jak wyżej, ale obliczający średnią arytmetyczną podanych liczb.

\zad Napisz program jak wyżej, ale obliczący najmniejszą i największą z podanych liczb.

\begin{center}
\includegraphics[width=0.8\textwidth]{zglupial.jpg}
\end{center}

\newpage\reset
\section*{Grupa starsza. Sito Erastotenesa}

\zad Na jakie pytanie odpowiada poniższy program? ({\tt n} to liczba typu {\tt int})
\begin{lstlisting}
int d = 2;
while (d * d <= n) {
	if (n % d == 0) break;
	d++;
}
if (d * d <= n) {
	cout << "nie" << endl;
} else {
	cout << "tak" << endl;
}
\end{lstlisting}

\noindent
Ile maksymalnie obrotów pętli wykona powyższy algorytm?

\zad Sito Erastotenesa to algorytm pozwalający szybko wyznaczyć liczby pierwsze z przedziału od $1$ do $n$ (szybko, to znaczy w czasie proporcjonalnym do $n \log \log n$. Iteracyjne wykorzystanie poprzedniej funkcji dałoby algorytm
o złożoności proporcjonalnej do $n \sqrt{n}$. Dla $n = 10^6$ mamy $\log \log n < 1$ a $\sqrt{n} = 1000$). 

\vspace{3mm}
Schemat algorytmu (sito Erastotenesa)

\begin{enumerate}
\item Utwórz tablicę {\tt bool p[n+1]} i wypełń ją wartościami {\tt true}.
\item { \tt p[0] = false; p[1] = false; }
\item { \tt d = 2; }
\item Dopóki {\tt d * d <= n} wykonuj następujące operacje
  \begin{enumerate}
  \item Dla {\tt k = 2, 3, $\ldots$} dopóki {\tt k * d <= n} podstawiaj {\tt p[k * d]  = false;}
  \item Znajdź następne d takie, że {\tt d*d <= n} i {\tt p[d] == true} 
  \end{enumerate}
\item Wypisz wszystkie liczby {\tt i}, dla których {\tt p[i] == true}.
\end{enumerate}

Algorytm ten wypisze wszystkie liczby pierwsze pomiędzy $2$ a $n$. Wyjaśnij jego działanie i zaimplementuj go.

\zad Napisz dwa programy, które obliczają ile jest liczb pierwszych od $2$ do $8000000$:

\pkt pierwszy wykorzystujący pierwszą metodę.

\pkt drugi wykorzystujący sito Erastotenesa.

\vspace{2mm}
Sprawdź czasy działania obydwu programów. Na moim komputerze jest to odpowiednio $5,21$ i $0,15$ sekundy.
  Oba algorytmy podają wynik $539777$.

\vspace{4mm}\noindent
{\bf Prawo Murphy'ego}

\vspace{2mm}
Jeżeli wydaje się, że wszystko działa dobrze, to z pewnością musiałeś coś przeoczyć.

\newpage
\begin{center}Szukanie dzielników\end{center}
\begin{lstlisting}
  int n; cin >> n;

  int ile = 0;

  for (int i = 2; i <= n; i++) {
    int d = 2;
    while (d * d <= i) {
      if (i % d == 0) break;
      d++;
    }
    if (d * d > i) { ile++;}
  }

  cout << ile << endl;
\end{lstlisting}

\vspace{5mm}
\begin{center}Sito Erastotenesa\end{center}
\begin{lstlisting}
  int n;  cin >> n;

  bool p[n+1];

  for (int i = 0; i <= n; i++) { p[i] = true; }

  p[0] = false; p[1] = false;

  int d = 2;

  while (d * d <= n) {
    for (int k = 2; d * k <= n; k++) {
      p[d * k] = false;
    }

    do {
      d++;
    } while (d * d <= n && p[d] == false);
  }

  int ile = 0;
  for (int i = 0; i <= n; i++) {
    if (p[i]) {
      ile++;
    }
  }
  cout << ile << endl;
\end{lstlisting}

\end{document}
